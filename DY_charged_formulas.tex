
\documentclass{article}
\usepackage{amsmath, amssymb}


\begin{document}

\title{W production differential cross section in transverse mass}
\author{}
\date{}
\maketitle

\section{Cross section}

I begin from the definition of the partonic cross section for W+ production from Eq. 2.2 in Ref.~\cite{CarloniCalame:2006zq}

\begin{equation}
 \frac{d\sigma_0}{d\Omega} (qq) = \frac{g^4 |V_{ud}|^2}{768\pi^2}\frac{1}{(\hat{s} - M_W^2)^2 + \Gamma_W^2 M_W^2} \frac{\hat{u}^2}{\hat{s}}
\end{equation}

With $\hat{u} = -\frac{\hat{s}}{2}(1+\cos\theta)$.

I convolute with the PDFs to get the hadronic cross section (I move the dependence on the CKM matrix elements in the piece which includes the PDFs, such that each initial state will be multiplied by the appropriate CKM matrix element).

\begin{equation}
 \frac{d\sigma_0}{dx_1 dx_2 d\Omega} (PP) = \sum_f \left[PDF(x_1) PDF(x_2) |V_{ud}|^2 \right]\frac{d\sigma_0}{d\Omega} (qq)
\end{equation}

From now on I identify $\hat{s} = M_{inv}^2$.

Usual change of integration variables from $(x_1,x_2)\to(Y,M_{inv}^2)$ with the Jacobian being $\frac{1}{s}$
(see Peskin pag. 565, below Eq. 17.47)

\begin{equation}
 \frac{d\sigma_0}{dY dM_{inv}^2 d\Omega} = \frac{d\sigma_0}{dx_1 dx_2 d\Omega} \frac{1}{s}
\end{equation}

Another change of integration variable from $Y\to Y_r$ with $Y=-\frac{1}{2}\log\left(\frac{M_{inv}^2}{s}\right)Y_r$ thus the Jacobian
$\frac{dY}{dY_r}=-\frac{1}{2}\log\left(\frac{M_{inv}^2}{s}\right)$

\begin{equation}
 \frac{d\sigma_0}{dY_r dM_{inv}^2 d\Omega} = \frac{d\sigma_0}{dY dM_{inv}^2 d\Omega} \left[-\frac{1}{2}\log\left(\frac{M_{inv}^2}{s}\right)\right]
\end{equation}


This change of variable is helpful for the integration in rapidity, in particular its integration limits: \\
$Y_r = 1 \to x_1 = \frac{M^2}{s}; x_2 = 1\\
Y_r = 0 \to x_1 = \sqrt{\frac{M_{inv}^2}{s}}; x_2 = \sqrt{\frac{M_{inv}^2}{s}}\\
Y_r = -1 \to x_1=1; x_2= \frac{M_{inv}^2}{s}$.

So the integration over $Y_r$ will be in the interval [-1,1].

Another change of variables to transverse mass $M_T=\sqrt{2p_T^{\ell} E_T^{miss} \left(1-\cos\left(\Delta \phi _{\ell \nu}\right)\right)}$ with $\Delta \phi _{\ell \nu }$ being the angle between the charged lepton and the neutrino in the transverse plane.

NOTE: now we assume the final state charged lepton and neutrino to be both massless: $p_T^{\ell} = E_T^{miss} = p_T$ with $p_T = M_{inv}\sin\theta$; furthermore at tree-level, the scattering happens orthogonal to the transverse plane, thus $\Delta \phi _{\ell \nu} = \pi$.
So, for the transverse mass we have $M_T =2 M_{inv}\sin\theta$ and $M_T^2= 4 M_{inv}^2 \left(1-\cos^2\theta \right)$.

This change of integration variable from $M_{inv}^2\to M_T^2$ with Jacobian $\frac{dM_{inv}^2}{dM_T^2}=\frac{1}{4\left(1-\cos^2\theta \right)}$.

\begin{equation}
 \frac{d\sigma_0}{dY_r dM_T^2 d\Omega} = \frac{d\sigma_0}{dY_r dM_{inv}^2 d\Omega}\frac{1}{4\left(1-\cos^2\theta \right)}
\end{equation}

Another change of integration variable from $M_T^2 \to M_T$:

\begin{equation}
 \frac{d\sigma_0}{dY_r dM_T d\Omega} = \frac{d\sigma_0}{dY_r dM_T^2 d\Omega} 2 M_T
\end{equation}

The cross section is independent on the azimuthal andle $\phi$, that we can immediately integrate upon:

\begin{equation}
 \frac{d\sigma_0}{dY_r dM_T d\cos\theta} = (2\pi)\frac{d\sigma_0}{dY_r dM_T d\Omega}
\end{equation}

The last convenient change of variable is between $\cos\theta$ and the pseudorapidity $\eta$.\\
$\eta = -\ln\left(\tan\frac{\theta}{2}\right)$ $\longrightarrow$ $\frac{d\cos\theta}{d\eta} = (1 - \cos^2\theta)$.

\begin{equation}
 \frac{d\sigma_0}{dY_r dM_T d\eta} = \frac{d\sigma_0}{dY_r dM_T d\cos\theta} (1 - \cos^2\theta)
\end{equation}

Since we need to integrate $\eta$ between plus and minus infinity, a convenient change of variables for the integration is $\eta = \frac{\eta_r}{1 - \eta_r^2}$ $\longrightarrow$ $\frac{d\eta}{d\eta_r} = \frac{1 + \eta_r^2}{(1 - \eta_r^2)^2}$:

\begin{equation}
 \frac{d\sigma_0}{dY_r dM_T d\eta_r} = \frac{d\sigma_0}{dY_r dM_T d\eta} \frac{1 + \eta_r^2}{(1 - \eta_r^2)^2}
\end{equation}

\section{Integration}

The integration is in the two variables $(Y_r, \eta_r)$.
The integration over $Y_r$ has been described in the previous section.

The integration over $\eta_r$ is done in the range [-1,1].
On the original variable $\eta$, we must apply the detection acceptance cut $|\eta| < \eta_{cut}$.
The cut is applied in the lab frame, so we have to boost the system accordingly.
The choice of $\eta$ is convenient because rapidities are additive under boost.
In the lab frame, the pseudorapidity is thus $\eta + Y$, so the cut is applied imposing: for $(\eta + Y < -\eta_{cut})$ or $(\eta + Y > \eta_{cut})$ $\longrightarrow$ $d\sigma = 0$.\\
% This means that the integration limits for $\eta$ (in the c.o.m. frame) have to be chosen carefully.
% It must be checked that for the maximum rapidity value $Y_{max}$ that is generated $|\eta_{limit}| > \eta_{cut} + Y_{max}$. This way one is sure to not cut away portions of the angular parameter space which contribute to the integral.

Also the cut on the $p_T$ is imposed in a similar way: for $p_T < p_{T_{cut}}$ $\longrightarrow$ $d\sigma = 0$.


% The integration over $\cos\theta$ is actually the integration over $\cos\theta^*$, meaning the angle is integrated in the c.o.m. frame.
% Our choice is to integrate in the c.o.m. frame.
% With this choice we are not transforming the angle from the c.o.m. frame to the lab frame according to the boost, but rather we transform the integration limits from the lab frame into the c.o.m. frame according to the boost.
% The integration limits are defined by the kinematic cuts in the lab frame.
% The integration without cuts would be in the interval [-1, 1].
% The kinematic cuts in the lab frame impose $\left|p_T\right| > 20 \text{GeV}$ and $\left|\eta _{\ell }\right| < 2.5$ with $\eta _{ \ell } = -\ln \left(\tan\frac{\theta }{2}\right)$.
% 
% The integration limits on $\cos\theta^*$ in the c.o.m. frame become:
% 
% \begin{align}
%  \cos\theta_{min} &= \max\left[-\tanh \left( \eta _{cut} - Y\right), -\sqrt{1-\frac{4 p_{T,cut}^2}{M_{inv}^2}} \right]\\
%  \cos\theta_{max} &= \min\left[\tanh \left( \eta _{cut} - Y\right), \sqrt{1-\frac{4 p_{T,cut}^2}{M_{inv}^2}} \right]\\
% \end{align}

% This procedure has been validated in the code for the neutral channel.


%%%%%%%%%%%%%%%%%%%%%%%%%%%%%%%%%%%%%%%%%%%%%%%%%%%%%%%%%%%%%%%%%%%%%%%%%%%%%%%%

\begin{thebibliography}{99}


%\cite{CarloniCalame:2006zq}
\bibitem{CarloniCalame:2006zq}
C.~M.~Carloni Calame, G.~Montagna, O.~Nicrosini and A.~Vicini,
%``Precision electroweak calculation of the charged current Drell-Yan process,''
JHEP \textbf{12} (2006), 016
doi:10.1088/1126-6708/2006/12/016
[arXiv:hep-ph/0609170 [hep-ph]].
%150 citations counted in INSPIRE as of 11 Nov 2020


\end{thebibliography}



\end{document}
